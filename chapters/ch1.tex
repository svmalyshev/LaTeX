\chapter{Возможности метода}

Масс-спектрометрия с иднуктивно связанной плазмой (ИСП) --- это метод анализа, широко применяемый в современной аналитической химии и измерения отношения \upb. Он основан на использовании иднуктивно связанной плазмы для ионизации ионов атомов или молекул в пробе, а затем их разделения на основе их масс-зарядового соотношения.

ИСП является одним из наиболее распространенных методов масс-спектрометрии, который    обладает высокой чувствительностью, точностью и разрешением. В рамках этого метода, проба вводится в индукционно связанную плазму, полученную путем инициирования газа высокой частоты. В результате, плазма достигает экстремально высокой температуры, что приводит к ионизации атомов и молекул пробы, сопровождающейся их фрагментацией.
