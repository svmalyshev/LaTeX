\documentclass{esdiploma}

\begin{document}

%% заполнение титульного листа
\filltitle{ru}{
    chair              = {Кафедра геологии всего сущего},
    title              = {Значение изучения вещества при изучении геосинклиналей в активных тектонических обстановках},
    % Здесь указывается тип работы. Возможные значения:
    %   coursework - Курсовая работа
    %   diploma - Диплом специалиста
    %   master - Диплом магистра
    %   bachelor - Диплом бакалавра
    type               = {master},
    author             = {Машкин Эдельвейс Захарович},
    supervisorPosition = {д.\,г.-м.\,н., профессор},
    supervisor         = {Выбегалло А.\,А.},
    reviewerPosition   = {с.н.с},
    reviewer           = {Складкин А.\,И.},
}

%% корректируем ссылки на источники типа объяснительных записок
\defcitealias{geolmap2011m-47}{Государственная..., 2011}

\maketitle
\tableofcontents

\newcommand{\upb}{$^{238}U/^{206}Pb$} % команда для задания сложных или длинных текстов


\renewcommand{\bibname}{СПИСОК ИСПОЛЬЗОВАННЫХ ИСТОЧНИКОВ}
\addcontentsline{toc}{chapter}{Список использованных источников}

%\providecommand*{\BibDash}{}
\providecommand*{\BibEmph}[1]{\emph{#1}}
%\providecommand*{\BibEmphi}[1]{\textbf{#1}}
%\providecommand*{\BibEmphii}[1]{\underline{#1}}
\makeatletter %http://tex.stackexchange.com/questions/40590/is-there-a-command-to-ignore-the-following-character
\def\?#1{}        % средство удаления последующего знака
\makeatother
\providecommand*{\BibUrl}[1]{\?}
\providecommand*{\url}[1]{\?}
\providecommand*{\BibDOI}[1]{#1}
%\providecommand*{\BibDash}{}
%\def\url#1{}

\bibliography{diploma.bib}


\end{document}
